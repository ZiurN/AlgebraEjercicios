\documentclass[a4paper,11pt]{article}
\usepackage[T1]{fontenc}				% Para que se puedan formar tildes como una sola letra
\usepackage[utf8]{inputenc}             % Para poder escribir tildes
\usepackage{textcomp}					% M\`as caracteres acentuados
\usepackage{multicol}					% Crea ambientes donde se puede escribir en varias columnas
\usepackage{amsmath,amssymb,amsfonts,latexsym,cancel}	% Paquete para la creaci\'on de ambientes y s\'imbolos matem\'aticos
\usepackage{amssymb}
%---------------------------------------------------------------------------------------------------------------------------
\usepackage{anysize}					% Para ajustar m\'argenes del documento
\marginsize{1cm}{1cm}{0.5cm}{0.3cm}
%\marginsize{L}{R}{U}{D}
\IfFileExists{enumitem.sty}{\usepackage{enumitem}}{}
%---------------------------------------------------------------------------------------------------------------------------
\usepackage{fancyhdr}					% Para editar encabezados y pies de p\'agina
\pagestyle{fancy}
\fancyfoot[C]{}
\fancyhead[r]{}
\fancyhead[l]{}
\fancyhead[c]{}
\renewcommand{\headrulewidth}{0pt} 			% grosor de la l\'inea de la cabecera
%---------------------------------------------------------------------------------------------------------------------------
\usepackage{color}					% Para generar y usar colores
\definecolor{azul}{rgb}{0,0,1}
\definecolor{g}{cmyk}{0.4,0.4,0.4,0.4}
%\definecolor{``nombre''}{``formato''}{0.0,0.0,0.0,0.0}
%---------------------------------------------------------------------------------------------------------------------------
\usepackage{graphicx}					% Para poder introducir gr\'aficas y fotos en formatos dif. a eps
\DeclareGraphicsExtensions{.pdf,.png,.jpg}
%---------------------------------------------------------------------------------------------------------------------------
\usepackage[colorlinks=true,backref=true,linkcolor=azul,citecolor=g,urlcolor=azul]{hyperref}
% Para insertar v\`inculos e hiperv\`iculos
%linkcolor=color (red) links internos
%citecolor=color (green) color de citaciones
%filecolor=color (magenta) ficheros locales
%pagecolor=color (red) otras p{\'a}ginas
%urlcolor=color (cyan) color de direcciones de internet url externas
%Otras opciones:
%backref=true para a{\~n}adir un link de retorno en la bibliograf{\'i}a .
%pagebackref=true lo mismo que el anterior pero ligado a la p{\'a}gina
%---------------------------------------------------------------------------------------------------------------------------
%---------------------------------------------------------------------------------------------------------------------------
\parindent 0mm						% Indexado o sangrado
%---------------------------------------------------------------------------------------------------------------------------
%T\`iTULO
\makeatletter
\renewcommand{\maketitle}{
\begin{center}
\begin{normalsize}\textbf{\@title}\end{normalsize}\\
\begin{normalsize}\@author\end{normalsize}
\end{center}
}

\usepackage[utf8]{inputenc}

\title{Álgebra - Pr\'actica 1 - Conjuntos, Relaciones y Funciones} 
\author{}
\date{1-2019}

\pdfinfo{%
  /Title    ()
  /Author   ()
  /Creator  ()
  /Producer ()
  /Subject  ()
  /Keywords ()
}
%-----------------Comandos creados para ayudar---------------
 \newcommand{\real}{\mathbb{R}}
%-------------------------------------------------------------
\begin{document}
\maketitle
\section*{Conjuntos}
    \begin{enumerate}
        \item Dado el conjunto $A=\{1,2,3\}$, determinar c\'uales de las siguientes afirmaciones son verdaderas:
        \begin{enumerate}[label = \roman*)]
            \item $1\in A$\\
                \textit{Verdadero}, 1 est\'a el la lista de elementos de $A$.
            \item $\{1\}\subseteq A$\\
                \textit{Verdadero}, el conjunto $\{1\}$ es subconjunto de $A$.
            \item $\{2,1\}\subseteq A$\\
                \textit{Verdadero}, en principio, los elementos de un conjunto no estan ordenados, luego $\{2,1\}=\{1,2\}$, y este conjunto es subconjunto de $A$.
            \item $\{1,3\}\in A$\\
                \textit{Falso}, el conjunto $\{1,3\}$ es \textit{subconjunto} de $A$, m\'as no est\'a en la lista de elementos de $A$.
            \item $\{2\}\in A$\\
                \textit{Falso}, Ocurre lo mismo que en el caso anterior.
        \end{enumerate}
        \item Dado el conjunto $A = \{1,2,\{3\},\{1,2\}\} $, determinar c\'uales de las siguientes afirmaciones son verdaderas:
        \begin{enumerate}[label = \roman*)]
            \item $3\in A$\\
                \textit{Falso}, El elemento de $A$ es el conjunto $\{3\}$, no el elemento $3$.
            \item $\{3\} \subseteq A$\\
                \textit{Falso}, el conjunto $\{3\}$ es \textit{un elemento} de $A$, no un subconjunto.
            \item $\{3\} \in A$\\
                \textit{Verdadero}.
            \item $\{\{3\}\} \subseteq A$\\
                \textit{Verdadero}, el conjunto de un \'unico elemento $\{\{3\}\}$ es subconjunto de $A$.
            \item $\{1,2\} \in A$\\
                \textit{Verdadero}.
            \item $\{1,2\} \subseteq A$\\
                \textit{Falso}, el conjunto $\{1,2\}$ es \textit{un elemento} de $A$.
            \item $\{\{1,2\}\} \subseteq A$\\
                \textit{Verdadero}, dado que $\{1,2\}$ es un elemento de $A$, el conjunto creado a partir de este \'unico elemento ser\'a subconjunto de $A$.
            \item $\{\{1,2\},3\} \subseteq A$\\
                \textit{Falso}, el elemento $3$ no pertenece a $A$, por lo tanto cualquier conojunto que tenga a $3$ no puede ser subconjunto de $A$.
            \item $\emptyset \in A$\\
                \textit{Falso}, el conjunto vacio es \textit{un subconjunto} de $A$.
            \item $\emptyset \subseteq A$\\
                \textit{Verdadero}.
            \item $A\in A$
                \textit{Falso}, Todo conjunto es subconjunto de s\'i mismo\footnote{Aunque es verdad que pueden darse casos en los que por la misma definici\'on del conjunto, este puede ser parte de s\'i mismo, No en este caso.}.
            \item $A\subseteq A$
                \textit{Verdadero}.
        \end{enumerate}
        \newpage
        \item Determinar si $A \subseteq B$ en cada uno de los siguientes casos:
            \begin{enumerate}[label = \roman*)]
                \item $A=\{1,2,3\}$, $B=\{5,4,3,2,1\}$\\
                    \textit{Verdadero}. los elementos 1, 2, 3 pertenecen a los dos conjuntos.
                 \item $A=\{1,2,3\}$, $B=\{1,2,\{3\},-3\}$\\
                    \textit{Falso}, ya que el elemento tres no hace parte de los elementos de $B$.
                \item $A=\{x\in \mathbb{R}/2<|x|<3\}$, $B=\{x\in \mathbb{R}/x^{2}<3\}$\\
                    \textit{Verdadero}, ya que los intervalos donde $x$ pertenece a $A$ son $(-3,-2)$ y $(2,3)$, mientras que el intevalo donde $x$ pertenece a $B$ es $(-3,3)$
                \item $A=\{\emptyset\}$, $B=\emptyset$
                    \textit{Falso}, ya que $B$ es el conjunto vacio, $B$ es subconjunto propio de cualquier otro subconjunto, m\'as no tiene ning\'un subconjunto propio.
            \end{enumerate}
        \item 
        \begin{enumerate}[label = \roman*)]
            \item Describir a los siguentes subconjuntos de $\real$ por comprensi\'on mediante \textit{una sola} ecuaci\'on:\\\\
            $\{-3,1,5\}$: 
            Aunque para describir este conjunto por compreni\'on hay varias maneras, una de las m\'as sencillas es expresar estos valores como el conjunto soluci\'on de las raices de un polinomio, asi, se puede describir el conjunto como:
            \begin{equation}
                \notag \{x\in \mathbb{R}/(x+3)(x-1)(x-5)=0\}
            \end{equation}
            \\\\
            $(-\infty,2]\cup[7,+\infty)$
        \end{enumerate}
    \end{enumerate}


\end{document}
